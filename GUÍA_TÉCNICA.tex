\documentclass[12pt,a4paper]{article}
\usepackage[utf8]{inputenc}
\usepackage[spanish]{babel}
\usepackage{hyperref}
\usepackage{enumitem}
\usepackage{geometry}
\geometry{margin=2.5cm}

\title{Guía Técnica de Programación\\SGUL - Sistema de Gestión Universitaria}
\author{Equipo de Desarrollo}
\date{\today}

\begin{document}

\maketitle

\section{Tecnologías y Arquitectura}
\begin{itemize}
    \item \textbf{Framework principal:} Django (Python)
    \item \textbf{Base de datos:} PostgreSQL (recomendado) o SQLite (para pruebas)
    \item \textbf{Frontend:} Templates HTML con Django Template Language (DTL), CSS básico, JS mínimo
    \item \textbf{Gestión de dependencias:} requirements.txt
    \item \textbf{Control de versiones:} Git
\end{itemize}

\section{Estructura del Proyecto}
\begin{verbatim}
sgul/
├── crm/                # Configuración global de Django (settings, urls, wsgi, asgi)
├── gestion/            # App principal: modelos, vistas, templates, urls, forms, etc.
│   ├── migrations/     # Migraciones de la base de datos
│   ├── templates/      # Archivos HTML organizados por vistas
│   ├── static/         # (opcional) Archivos estáticos propios
│   ├── models.py       # Modelos de datos (solo para autenticación y mínimos)
│   ├── views.py        # Lógica de vistas (muchas usando SQL puro)
│   ├── urls.py         # Rutas de la app
│   ├── forms.py        # Formularios personalizados
│   ├── decorators.py   # Decoradores para permisos y roles
│   └── ...             # Otros archivos auxiliares
├── manage.py           # Comando principal de Django
├── requirements.txt    # Dependencias del proyecto
└── README.md           # Guía de instalación y uso
\end{verbatim}

\section{Decisiones de Programación}
\begin{enumerate}[label=\alph*)]
    \item \textbf{Uso de SQL puro:} Muchas vistas usan consultas SQL directas (\texttt{cursor.execute(...)}) en vez del ORM de Django. Los resultados se transforman a diccionarios para facilitar su uso en los templates. Se cuida la seguridad usando parámetros en las consultas.
    \item \textbf{Roles y permisos:} Se implementan roles (Superadmin, Admin, Ventas, etc.) usando un campo personalizado en el modelo de usuario. Se usan decoradores para restringir el acceso a vistas según el rol. Los menús y enlaces en los templates se muestran/ocultan según el rol.
    \item \textbf{Templates y navegación:} Se usa un template base (\texttt{base.html}) con bloques para heredar. El menú de navegación se adapta dinámicamente según el rol. Se usan templates separados para cada funcionalidad.
    \item \textbf{Dashboards y reportes:} El dashboard principal muestra KPIs y estadísticas usando consultas SQL agregadas. El superadmin tiene un dashboard especial con más métricas y accesos rápidos.
    \item \textbf{Exportación/Importación de datos:} Se implementan vistas para exportar datos a CSV/XLSX y para importar desde archivos. Se usa la librería \texttt{openpyxl} para manejar archivos Excel.
    \item \textbf{Consultas SQL personalizadas:} El superadmin puede ejecutar cualquier consulta SQL desde el panel, con manejo de errores y mensajes de éxito.
    \item \textbf{Visualización de la base de datos:} Hay una vista especial para ver todas las tablas, columnas, claves foráneas y estadísticas de la base de datos.
\end{enumerate}

\section{Flujo de Desarrollo}
\begin{enumerate}
    \item \textbf{Modelado inicial:} Se definieron los modelos mínimos en Django para autenticación y roles. El resto de la lógica se maneja con SQL puro.
    \item \textbf{Migraciones y carga de datos:} Se usaron migraciones de Django para crear las tablas base y scripts SQL para poblar datos iniciales.
    \item \textbf{Vistas y templates:} Se crearon vistas para cada funcionalidad (leads, matrículas, clientes, dashboards, etc.). Se adaptaron los templates para trabajar con resultados de SQL.
    \item \textbf{Roles y permisos:} Se implementaron decoradores y lógica en los templates para restringir accesos.
    \item \textbf{Dashboards y reportes:} Se programaron consultas SQL agregadas para mostrar estadísticas y KPIs.
    \item \textbf{Herramientas de administración:} Exportación/importación de datos, ejecución de consultas SQL personalizadas, visualización de la estructura de la base de datos.
    \item \textbf{Pruebas y ajustes:} Se probaron los flujos principales con distintos roles y se corrigieron errores de integración.
\end{enumerate}

\section{Buenas Prácticas y Seguridad}
\begin{itemize}
    \item Se usan parámetros en las consultas SQL para evitar inyecciones.
    \item Los archivos sensibles (contraseñas, claves) no se suben al repo.
    \item Se recomienda usar variables de entorno para credenciales en producción.
    \item Se documenta todo en el README y en los comentarios del código.
\end{itemize}

\section{¿Cómo contribuir?}
\begin{enumerate}
    \item Haz un fork o clona el repo.
    \item Crea una rama para tu funcionalidad.
    \item Haz tus cambios y pruebas.
    \item Haz un pull request describiendo tu aporte.
\end{enumerate}

\section{Notas finales}
El sistema está pensado para ser fácilmente extensible: puedes agregar nuevas vistas, templates o reportes siguiendo la misma lógica. Si tienes dudas sobre alguna consulta SQL o flujo, revisa los comentarios en las vistas o pregunta al responsable técnico.

\end{document} 